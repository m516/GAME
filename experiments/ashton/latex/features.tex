\index{features@{features}} 
\begin{DoxyItemize}
\item Requires very little overhead from the writer of the documentation. Plain text will do, Markdown is support, and for more fancy or structured output H\+T\+ML tags and/or some of doxygen\textquotesingle{}s special commands can be used. 
\item Cross platform\+: works on Windows and many Unix flavors (including Linux and Mac\+O\+SX). 
\item Indexes, organizes and generates browsable and cross-\/referenced output even from undocumented code. 
\item Generates structured X\+ML output for parsed sources, which can be used by external tools. 
\item Supports C/\+C++, Java, (Corba and Microsoft) Java, Python, V\+H\+DL, P\+HP I\+DL, \mbox{\hyperlink{class_c}{C}}\#, Fortran, T\+CL, Objective-\/C 2.\+0, and to some extent \mbox{\hyperlink{class_d}{D}} sources. 
\item Supports documentation of files, namespaces, packages, classes, structs, unions, templates, variables, functions, typedefs, enums and defines. 
\item Javadoc (\mbox{\hyperlink{namespace1_1_11}{1.\+1}}), qdoc3 (partially), and E\+C\+M\+A-\/334 (\mbox{\hyperlink{class_c}{C}}\# spec.) compatible. 
\item Comes with a G\+UI frontend (Doxywizard) to ease editing the options and run doxygen. The G\+UI is available on Windows, Linux, and Mac\+O\+SX. 
\item Automatically generates class and collaboration diagrams in H\+T\+ML (as clickable image maps) and \textbackslash{}\+La\+TeX (as Encapsulated Post\+Script images). 
\item Uses the {\ttfamily dot} tool of the Graphviz tool kit to generate include dependency graphs, collaboration diagrams, call graphs, directory structure graphs, and graphical class hierarchy graphs. 
\item Allows grouping of entities in modules and creating a hierarchy of modules. 
\item Flexible comment placement\+: Allows you to put documentation in the header file (before the declaration of an entity), source file (before the definition of an entity) or in a separate file. 
\item Generates a list of all members of a class (including any inherited members) along with their protection level. 
\item Outputs documentation in on-\/line format (X\+H\+T\+ML and U\+N\+IX man page) and off-\/line format (\textbackslash{}\+La\+TeX and R\+TF) simultaneously (any of these can be disabled if desired). All formats are optimized for ease of reading. ~\newline
 Furthermore, compressed H\+T\+ML can be generated from H\+T\+ML output using Microsoft\textquotesingle{}s H\+T\+ML Help Workshop (Windows only) and P\+DF can be generated from the \textbackslash{}\+La\+TeX output. 
\item Support for various third party help formats including H\+T\+ML Help, docsets, Qt-\/\+Help, and eclipse help. 
\item Includes a full \mbox{\hyperlink{class_c}{C}} preprocessor to allow proper parsing of conditional code fragments and to allow expansion of all or part of macros definitions. 
\item Automatically detects public, protected and private sections, as well as the Qt specific signal and slots sections. Extraction of private class members is optional. 
\item Automatically generates references to documented classes, files, namespaces and members. Documentation of global functions, global variables, typedefs, defines and enumerations is also supported. 
\item References to base/super classes and inherited/overridden members are generated automatically. 
\item Includes a fast, rank based search engine to search for strings or words in the class and member documentation (P\+HP based). 
\item Includes an Javascript based live search feature to search for symbols as you type (for small to medium sized projects). 
\item You can type normal H\+T\+ML tags in your documentation. \mbox{\hyperlink{class_doxygen}{Doxygen}} will convert them to their equivalent \textbackslash{}\+La\+TeX, R\+TF, and man-\/page counterparts automatically. 
\item Allows references to documentation generated for other (doxygen documented) projects (or another part of the same project) in a location independent way. 
\item Allows inclusion of source code examples that are automatically cross-\/referenced with the documentation. 
\item Inclusion of undocumented classes is also supported, allowing to quickly learn the structure and interfaces of a (large) piece of code without looking into the implementation details. 
\item Allows automatic cross-\/referencing of (documented) entities with their definition in the source code. 
\item All source code fragments are syntax highlighted for ease of reading. 
\item Allows inclusion of function/member/class definitions in the documentation. 
\item All options are read from an easy to edit and (optionally) annotated configuration file. 
\item Documentation and search engine can be transferred to another location or machine without regenerating the documentation. 
\item Supports many different character encodings and uses U\+T\+F-\/8 internally and for the generated output. 
\item \mbox{\hyperlink{class_doxygen}{Doxygen}} can generate a layout which you can use and edit to change the layout of each page. 
\item There more than a 100 configurable options to fine-\/tune the output. 
\item Can cope with large projects easily. 
\end{DoxyItemize}

Although doxygen can now be used in any project written in a language that is supported by doxygen, initially it was specifically designed to be used for projects that make use of Qt Software\textquotesingle{}s \href{https://www.qt.io/developers/}{\texttt{ Qt toolkit}}. I have tried to make doxygen `Qt-\/compatible'. That is\+: \mbox{\hyperlink{class_doxygen}{Doxygen}} can read the documentation contained in the Qt source code and create a class browser that looks quite similar to the one that is generated by Qt Software. \mbox{\hyperlink{class_doxygen}{Doxygen}} understands the C++ extensions used by Qt such as signals and slots and many of the markup commands used in the Qt sources.

\mbox{\hyperlink{class_doxygen}{Doxygen}} can also automatically generate links to existing documentation that was generated with doxygen or with Qt\textquotesingle{}s non-\/public class browser generator. For a Qt based project this means that whenever you refer to members or classes belonging to the Qt toolkit, a link will be generated to the Qt documentation. This is done independent of where this documentation is located!



... 